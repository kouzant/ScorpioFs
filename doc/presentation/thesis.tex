\documentclass{beamer}
\usepackage{xgreek}
\usepackage{xltxtra}
\usepackage{graphicx}
\useoutertheme{shadow}
\usetheme{Singapore}
\setsansfont[Mapping=tex-text]{GFS Didot}
\setmonofont[Mapping=tex-text]{DejaVu Sans Mono}
%Remove navigation symbols
\setbeamertemplate{navigation symbols}{}
%effect so overlays not yet revealed will faintly appear
\setbeamercovered{dynamic}

\title[ScorpioFS]{ScorpioFS\\Κατανεμημένο ομότιμο σύστημα αρχείων}
\author{Αντώνης Κουζούπης}
\institute{Πανεπιστήμιο Πειραιώς\\Τμήμα Πληροφορικής}
\date{\today}
\logo{\includegraphics[scale=0.5]{unipi_logo.jpg}}

\begin{document}
\frame{
    \titlepage
}
\section{Εισαγωγή}
\subsection{}
\frame{
\frametitle{Τι είναι το ScorpioFS}
\begin{itemize}
\item Σύστημα αποθήκευσης αντιγράφων ασφαλείας \\ \footnotesize 
Παρέχει στο χρήστη ένα τοπικά προσαρτημένο σύστημα αρχείων το οποίο αποθηκεύει
τα περιεχόμενά του στο δίκτυο.\normalsize
\pause
\item Δίκτυο ομότιμα συνδεδεμένων υπολογιστών \\
\footnotesize Αποτελεί ένα δίκτυο υπολογιστών που παρέχουν
στην υπηρεσία μία τοπική αποθήκη καθώς και μία λειτουργία αντιγραφής των αρχείων
μεταξύ των κόμβων για την εξασφάλιση της ακεραιότητας των δεδομένων. Όλοι οι κόμβοι 
στο δίκτυο είναι ομότιμα συνδεδεμένοι (peer--to--peer).\normalsize
\pause
\item Κατανεμημένο σύστημα \\
\footnotesize Το σύστημα αποθήκευσης αρχείων είναι πλήρως αποκεντρωμένο. Όλοι οι
κόμβοι στο δίκτυο έχουν ισότιμα δικαιώματα. Κληρονομεί τα πλεονεκτήματα και τα
μειονεκτήματα των κατανεμημένων συστημάτων.\normalsize
\end{itemize}
}
\subsection{}
\frame{
\frametitle{Τα μέρη του ScorpioFS \\ (Chord)(Σκατά Τίτλος!!!)}
%\transboxin
Το μέρος του \emph{ScorpioFS} που υλοποιεί το
Chord πρωτόκολλο. Είναι υπεύθυνο για την εύρεση των κόμβων που είναι
αποθηκευμένα τα δεδομένα, την εισαγωγή και τη διαγραφή ενός κόμβου από το δίκτυο
και για την αντιγραφή των αρχείων. Γενικά είναι υπεύθυνο για το δικτυακό
κομμάτι.
}

\frame{
\frametitle{Τα μέρη του ScorpioFS \\ (Fuse)(Σκατά Τίτλος!!!)}
Υλοποιεί το τοπικό σύστημα αρχείων που αντιλαμβάνεται ο
χρήστης. Υλοποιεί τις περισσότερες λειτουργίες ενός συστήματος αρχείων όπως
δημιουργία, διαγραφή, επεξεργασία, αντιγραφή κτλ. Χωρίζει μεγάλα αρχεία σε
μικρότερα του 1MB και επικοινωνεί με το \textbf{Chord} κομμάτι για την αποστολή
και αποδοχή δεδομένων.
}

\frame{
\frametitle{Τα μέρη του ScorpioFS \\ (Console)(Σκατά Τίτλος!!!)}
Κονσόλα διαχείρισης των κόμβων του δικτύου. Εκτελεί διάφορες λειτουργίες μαζικά
στους κόμβους όπως δημιουργία ή καταστροφή, περισυλλογή των στατιστικών.
Λειτουργεί ανεξάρτητα από το \textbf{Chord} και \textbf{Fuse} κομμάτι και
επιτελεί επικουρικό ρόλο στο σύστημα.
}
\section{Chord}
\section{FUSE}
\section{ScorpioFS}
\section{Πειράματα}
\section{Εκτέλεση}
\section{Επίδειξη}
\section{Μελλοντική Εργασία}
\end{document}
