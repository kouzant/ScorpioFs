\documentclass{beamer}
\usepackage{xgreek}
\usepackage{xltxtra}
\usepackage{graphicx}
\useoutertheme{shadow}
\usetheme{Singapore}
\setsansfont[Mapping=tex-text]{GFS Didot}
\setmonofont[Mapping=tex-text]{DejaVu Sans Mono}
%Remove navigation symbols
\setbeamertemplate{navigation symbols}{}
%effect so overlays not yet revealed will faintly appear
\setbeamercovered{dynamic}

\title[ScorpioFS]{ScorpioFS\\Κατανεμημένο ομότιμο σύστημα αρχείων}
\author{Αντώνης Κουζούπης}
\institute{Πανεπιστήμιο Πειραιώς\\Τμήμα Πληροφορικής}
\date{\today}
\logo{\includegraphics[scale=0.5]{unipi_logo.jpg}}

\begin{document}
\frame{
    \titlepage
}
\section{Εισαγωγή}
\subsection{}
\frame{
\frametitle{Τι είναι το ScorpioFS}
\begin{itemize}
\item Σύστημα αποθήκευσης αντιγράφων ασφαλείας \\ \footnotesize 
Παρέχει στο χρήστη ένα τοπικά προσαρτημένο σύστημα αρχείων το οποίο αποθηκεύει
τα περιεχόμενά του στο δίκτυο.\normalsize
\pause
\item Δίκτυο ομότιμα συνδεδεμένων υπολογιστών \\
\footnotesize Αποτελεί ένα δίκτυο υπολογιστών που παρέχουν
στην υπηρεσία μία τοπική αποθήκη καθώς και μία λειτουργία αντιγραφής των αρχείων
μεταξύ των κόμβων για την εξασφάλιση της ακεραιότητας των δεδομένων. Όλοι οι κόμβοι 
στο δίκτυο είναι ομότιμα συνδεδεμένοι (peer--to--peer).\normalsize
\pause
\item Κατανεμημένο σύστημα \\
\footnotesize Το σύστημα αποθήκευσης αρχείων είναι πλήρως αποκεντρωμένο. Όλοι οι
κόμβοι στο δίκτυο έχουν ισότιμα δικαιώματα. Κληρονομεί τα πλεονεκτήματα και τα
μειονεκτήματα των κατανεμημένων συστημάτων.\normalsize
\end{itemize}
}
\subsection{}
\frame{
\transboxin
sdkf
}
\section{Chord}
\section{FUSE}
\section{ScorpioFS}
\section{Πειράματα}
\section{Εκτέλεση}
\section{Επίδειξη}
\section{Μελλοντική Εργασία}
\end{document}
