\documentclass[a4paper]{article}
\usepackage{xgreek}
\usepackage{xunicode}
\usepackage{xltxtra}
\usepackage{listings}
\setlength{\topmargin}{0in}
\setlength{\oddsidemargin}{0in}
\setlength{\evensidemargin}{0in}
\setlength{\textheight}{9in}
\setlength{\textwidth}{6.25in}
\setmainfont[Mapping=tex-text]{DejaVu Serif}

\begin{document}
\section{Κρυπτογράφηση}
Τα αρχεία του χρήστη στην περίπτωση που είναι μεγαλύτερα από 1MB τότε ``σπάνε''
σε αρχεία του 1MB. Όταν έχουμε ένα αρχείο 50MB τότε θα δημιουργηθούν 50
μικρότερα αρχεία (chunks) το ενός MegaByte. Για να αναδομηθεί σωστά το αρχείο,
θα πρέπει και τα 50 chunks να μπουν στη σωστή σειρά. Επομένως έχουμε πάρα
πολλούς συνδυασμούς (πόσους???) για να πάρουμε με τη σωστή σειρά τα chunks.

Όλη την πληροφορία για το πως θα ενωθούν τα chunks για να φτιάξουν το αρχικό
αρχείο και γενικά τη δομή του προσαρτημένου καταλόγου την έχει το αρχείο με τα
μετα-δεδομένα (meta-data). Κάποιος κακόβουλος χρήστης, αν αποκτήσει το
συγκεκριμένο αρχείο, θα μπορέσει να αναδομήσει όλο τον προσαρτημένο κατάλογο.
Λόγω της φύσης των δεδομένων που αποθηκεύονται στο αρχείο αυτό, δύσκολα να
ξεπεράσει σε μέγεθος τα 10MB. Αυτό σημαίνει ότι θα ``σπάσει'' σε 10 μικρότερα
αρχεία. Ακόμα και έτσι είναι αρκετά εύκολο για κάποιον να φτιάξει το αρχικό
αρχείο που θα του δώσει τον προσαρτημένο κατάλογο.

Για τον παραπάνω λόγο κρίθηκε αναγκαία η κρυπτογράφηση του αρχείου με τα
meta-data. Οι αλγόριθμοι που χρησιμοποιήθηκαν καθώς και ο τρόπος που τους
υλοποιεί η Java παρουσιάζονται παρακάτω.

\subsection{Παραγωγή κλειδιού}
Για την κρυπτογράφηση είναι απαραίτητη η δημιουργία ενός κλειδιού
κρυπτογράφησης. Συγκεκριμένα χρησιμοποιήθηκε η PBKDF2 (Password-Based Key
Derivation Function) για την παραγωγή του κλειδιού. Για την παραγωγή του MAC
(Message Authentication Code) χρησιμοποιήθηκε ο αλγόριθμος HMAC με αλγόριθμο
κατακερματισμού SHA-1

Για αλγόριθμος κρυπτογράφησης χρησιμοποιήθηκε ο AES με μέγεθος κλειδιού 128
bit, σε CBC (Cipher-Block Chaining) mode και για padding αυτό που ορίζει το
standard PKCS\#5 (Public-Key Cryptography Standards νούμερο 5).

\subsection{Υλοποίηση σε Java}
Αρχικά για την παραγωγή του κλειδιού θα πρέπει να δημιουργήσουμε ένα byte array
για την αποθήκευση του salt όπως φαίνεται παρακάτω:
\lstset{language=Java, numbers=left}
\lstinputlisting{sourceCode/Salt.java}

Στη συνέχεια για τη δημιουργία του κλειδιού δημιουργείται ένα αντικείμενο τύπου
\emph{SecretKeyFactory}. Έπειτα δημιουργούμε τις προδιαγραφές του κλειδιού, δηλαδή τον
κωδικό που δίνει ο χρήστης, το salt που χρειάζεται και το μέγεθος του κλειδιού.
Τέλος παίρνουμε το αντικείμενο \emph{secret} τύπου \emph{SecretKey} που είναι το
τελικό μας κλειδί. Για τον αλγόριθμο κρυπτογράφησης αρκεί μία γραμμή για να
πάρουμε το αντικείμενο \emph{cipher}. Όπως φαίνεται παρακάτω δίνουμε σαν όρισμα
τις προδιαγραφές του αλγόριθμου κρυπτογράφησης και τον provider των μεθόδων
κρυπτογράφησης Στην συγκεκριμένη περίπτωση χρησιμοποιούμε τον Bouncy Castle. 
\lstset{language=Java, numbers=left}
\lstinputlisting{sourceCode/KeynCipher.java}

Έως αυτό το σημείο, τα βήματα είναι τα ίδια για την κρυπτογράφηση και την
αποκρυπτογράφηση.

\subsubsection{Κρυπτογράφηση}
Για την κρυπτογράφηση, αρχικοποιούμε το αντικείμενο \emph{cipher} για
κρυπτογράφηση Για την αποκρυπτογράφηση χρειάζεται το salt και το initialization
vector, οπότε και τα γράφουμε σε δύο αρχεία. Τέλος με τις μεθόδους
\emph{cipher.update()} και \emph{cipher.doFinal()} κρυπτογραφούμε όσα bytes
διαβάζουμε και τα γράφουμε σε ένα άλλο αρχείο.
\lstset{language=Java, numbers=left}
\lstinputlisting{sourceCode/Encrypt.java}

\subsubsection{Αποκρυπτογράφηση}
Για την αποκρυπτογράφηση χρειαζόμαστε το salt και το initialization vector που
χρησιμοποιήθηκαν κατά την κρυπτογράφηση οπότε και τα διαβάζουμε από τα αρχεία
που τα είχαμε αποθηκεύσει. Έπειτα αρχικοποιούμε το αντικείμενο \emph{cipher} για
αποκρυπτογράφηση, διαβάζουμε τα bytes από το κρυπτογραφημένο αρχείο και πάλι με
τις μεθόδους \emph{cipher.update()} και \emph{cipher.doFinal()} τα
αποκρυπτογραφούμε και τα γράφουμε σε ένα άλλο.
\lstset{language=Java, numbers=left}
\lstinputlisting{sourceCode/Decrypt.java}
\end{document}
